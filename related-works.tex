\section{Related Work}
\label{sec:related_works}

We addressed the main architectural concerns of SISDOT 
using well-known patterns for enterprise
systems~\cite{enterprise-patterns:book} and
software architecture patterns~\cite{pattern-oriented:book}.
Moreover, the ideas of using rule based engines to support
inference within software systems are also not new~\cite{ORDONEZ2016353,li2012modeling},
and have been particularly explored in the health
domain~\cite{mantas2012comparing,li2012modeling,jung2011executing}.
For instance, Van Hille et al. present a comparison
between Drools and OWL + SWRL for implementing the reasoning
procedures of the alert module of an implantable
cardioverter defibrillator. The authors conclude that
Drools provides greater expressiveness when compared to the other
approach. Our decision to use Drools to implement
low-level rules was mostly based on its integration to
the JEE platform, instead of its expressiveness.
We also considered the use of Prolog for at least
expressing the mechanics of the QDM generation, though
we postponed this work to a future work.

Dingcheng Li and colleagues report on the use of the
Drools engine to model and reason about \emph{electronic
  health records}~\cite{li2012modeling}. To this end,
the authors of this mentioned work implemented a
\emph{model-to-model translational approach} that
converts the specification of phenotyping algorithms 
expressed using the Quality Data Model (from the National
Quality Forum\footnote{http://www.qualityforum.org/Home.aspx})
into Drools rules. Similarly, Jung et al. present an approach
for executing medical logic modules expressed in ArdenML
using Drools~\cite{jung2011executing}. They also used a model-to-model appraoch
implemented using XSLT. Our approach could also be characterized
as being based on \emph{model-to-model transformations}, since we translate
a business entity model (represented as the \callers Java
domain objects in runtime) into Drools rules, though using FreeMarker as 
template engine. The work of Ostermayer et al. also advocates
the use of DSLs + template engines to generate Drools
rules~\cite{toostermayer2013simplifying}.

