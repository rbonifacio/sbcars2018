\section{Introduction}\label{sec:Introduction}

% Uma das fases da Logística de Materiais de Emprego Militar das Forças Armadas contempla o planejamento da distribuição de materiais para todas as unidades militares. Essa fase é conhecida como Dotação de Materiais, e envolve a catalogação de materiais, muitas vezes independente de fornecedor, a definição de regras que associam materiais de emprego militar a unidades organizacionais (por exemplo, diretorias, departamentos, batalhões e cargos militares) e a execução das regras para derivar os materiais inicialmente previstos para cada unidade militar e os materiais que devem ser realmente distribuídos para as organizações militares, os quais podem sofrer variações dadas as inclusões e supressões de cargos de uma determinada unidade organizacional do Exército Brasileiro.

One of the logistics' phases to distribute equipments to an army force contemplates the \emph{distribution planning} of materials throughout military units. This phase is known as \emph{material endowment}, and involves several steps, including the specification of 
relevant materials, often independent of vendor; the definition of rules that link military materials and equipments to organizational units (e.g., departments, regiments, and military posts); the execution of the rules for deriving the materials originally intended for each \emph{abstract military unit}; and the designation of the materials that must be actually distributed to the \emph{concrete military organizations}, which may suffer variations due to the inclusions and suppressions of positions for a particular organizational unit of an army force. In the context of this paper, the reader might consider the scenario of the Brazilian Army. 
 
% Sistemas de software de apoio à decisão tornam-se essenciais neste contexto, dada à importância e complexidade dos processos relacionados à Dotação de Materiais de Emprego Militar. Particularmente, as regras usadas para derivar à previsão de material e material previsto não são triviais, e quando implementadas diretamente no código fonte, tornam o sistema difícil de compreender e manter, uma vez que a quantidade de condições que precisam ser expressas é  proporcional à estrutura das unidades organizacionais. Além disso, uma solução "\emph{hard-coding}" não permite aos especialistas do domínio especificarem, de forma trivial, novas regras de derivação. Por essa razão, a externalização das regras de derivação voltadas para a dotação de material é preferencialmente implementada de forma declarativa (e não operacionalmente em uma linguagem de programação como C++ ou Java), usando linguagens, ferramentas ou bibliotecas que suportam algum motor de regras de negócio (\emph{business rule engines}).

Decision support systems are essential in this domain, given the importance and complexity of the processes related to the allocation of military employment materials. In particular, the rules used to predict military materials and equipments are not trivial, and implementing these rules directly in the source code makes the systems difficult to understand and maintain. This problem mostly occurs because the number of conditions that need to be expressed is proportional to the complexity of the organizational structure and the number of materials. In the case of an entire army force, this number is noteworthy. In addition, a "\emph{hard-coding}" solution does not allow domain experts to trivially specify and test new derivation rules. For these reasons, it is preferably to implement these rules declaratively, using languages, tools, or libraries that support some sort of inference mechanism.

% Por outro lado, a introdução de linguagens como Prolog~\cite{clocksin2003programming} ou bibliotecas como  Drools~\cite{amador2012}, \cite{bali2009}, \cite{bali2013}, \cite{browne2009} em ambientes corporativos pode sofrer algum tipo de rejeição, particularmente em ambientes cuja arquitetura de sistemas corporativos é bem estabelecida. Além disso, essas soluções requerem uma curva razoável de  aprendizagem, o que pode prejudicar atividades de manutenção de software. 

Nevertheless, introducing either a logic language (e.g., Prolog or Datalog) or a specific library (such as  Drools~\cite{browne2009}) in a complex enterprise system may lead to some type of \emph{architectural 
conflict}, particularly in environments whose whole ecosystem system is well established and based on a set of 
constraints related to both language and libraries usage. Furthermore, these solutions require a reasonable learning curve, which can hamper software maintenance activities.

% Esse artigo descreve as decisões arquiteturais adotadas para implementar o mecanismo de derivação de materiais e materiais previstos para o sistema de dotação de materiais do Exército Brasileiro, que objetiva abstrair completamente o uso de um motor de regras de negócio e permitir que especialistas do domínio simulem a definição e testes de novas regras---usando, na essência, técnicas de programação generativa (como linguagens específicas do domínio e meta-programação). Dessa forma, este artigo apresenta as seguintes contribuições:

This article describes the main design and architectural decisions adopted to implement the mechanisms that assist 
decision makers to plan the distribution of military materials and equipments across the organizational structure 
of the Brazilian Army. These decisions aim to both (a) abstract the adoption of a business rule engine using {\bf metaprogramming} 
and (b) allow domain experts to simulate the definition and testing of rules using 
a {\bf domain specific language}. Accordingly, this paper presents the following contributions:

\begin{itemize}
%   \item O uso de técnicas de programação generativa para o projeto arquitetural e implementação do mecanismo de derivação de materiais e materiais previstos para unidades organizacionais.
\item A description of the use of generative programming techniques to raise the abstraction level 
 related to the adoption of a logic programming language in the specification of rules for the 
 distribution of materials and equipments through organizational units.

% \item O relato da análise da arquitetura proposta, que reforça que as decisões arquiteturais tomadas foram adequadas e atendem às necessidades dos especialistas do domínio e às restrições técnicas do ambiente corporativo, gerando automaticamente as regras de distribuição de materiais.
\item A report of an empirical evaluation of our design and architectural decisions, reinforcing that they fulfill 
 not only the needs of the domain specialists, but also existing technical constraints. 

\end{itemize} 

% Apesar de discutirmos a distribuição de materiais para o domínio militar (um domínio não trivial), acreditamos que as decisões arquiteturais discutidas nesse artigo possam ser exploradas para outros contextos relacionados à distribuição de materiais e cenários em que se deseja introduzir motores de regras de negócio de forma mais transparente em sistemas corporativos. 

Although we discuss the distribution of materials and equipments for the military domain (which is far from trivial), we believe that the architectural decisions discussed here can be exploited to other logistics and endorsements scenarios in which it is also desirable to transparently introduce the support for rule based engines in enterprise systems.

% Este artigo está organizado como segue. Na Seção \ref{sec:Theory} é apresentado os conceitos relativos a logística de material, arquitetura de software, rule engine e programação generativa que são necessários para o entendimento deste trabalho. A Seção \ref{sec:architecture} apresenta os requisitos arquiteturais e de negócio da solução proposta, bem como os detalhes da implementação. A Seção \ref{sec:case_study} apresenta o estudo de caso desenvolvido para validar a proposta e por fim, a Seção \ref{sec:conclusao} apresenta as conclusões deste trabalho, bem como os trabalhos futuros. 

We organized this article as follows. Section \ref{sec:logistics} presents the concepts 
related to the domain of material logistics. Section~\ref{sec:rbs} presents the concepts 
related to business rules and rule based engines and Section~\ref{sec:gp} presents some 
background information related to \emph{generative programming}---techniques that we use to support our 
design and implementation. Section~\ref{sec:architecture} presents the design and architectural decisions we take 
to develop the mechanism responsible for the distribution of materials. In Section~\ref{sec:case_study} we present the 
results of an empirical study that validate our approach. Finally, Section~\ref{sec:conclusao} presents some final 
remarks. 

