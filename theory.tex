\section{Logistics Systems}
\label{sec:logistics}

Logistics is the method used to implement military strategies and tactics as a method to gain 
competitive advantage~\cite{rutner2012}. Logistics is defined as the art of moving armies, 
as well as accommodating and supplying the military~\cite{prebilic2006}. Strategy and tactics 
provide the framework for conducting military operations, while logistics provides the means.
In the context of this paper, we are more interested in the supplying aspects of logistics. 
The logistics supplying function refers to the set of activities that deals with the  provisioning of 
the material of different classes (including vehicles, armaments, munition, uniform, and fuel) necessary to the 
military unities (including a rich organizational structure and a huge number of soldiers). 

In order to facilitate the administration and control of materials and equipment, the Brazilian Army uses a 
catalog system that is similar to one defined by the North Atlantic Treaty Organization (OTAN)~\cite{otan2012}, 
which classifies items into groups and classes. Instead, to support the provisioning of materials and equipment, 
the Brazilian Army catalog uses a classification that considers the \emph{type}, \emph{class}, and \emph{family} of the 
materials, ignoring some details (such as specific brands and suppliers). More detailed information is considered by 
other areas of the logistic system. Several rules govern the identification of materials and equipment, 
to avoid duplication and simplify the process of decision making. A logistic system is the integrated set of organizations, 
personnel, principles, and technical standards intended to provide the adequate flow for supplying materials~\cite{brasil2003}. 
This is a complex system, and thus must be organized in different modules. 

In this paper we present the main design and architectural decisions related to one of the modules (SISDOT) of the \emph{Integrated 
Logistics System} (SIGELOG) of the Brazilian Army. SISDOT deals with the identification of the necessary 
materials and equipment (hereafter MEMs) for the organizational units of the Brazilian Army. Each organizational 
unity is a combination of a set of \emph{generic organizations of specific types}, which describes the expected 
number of military unities of an organization (including armies, divisions, brigades, and so on). The goal of 
SISDOT is to estimate the set of MEMs that should be assigned to a generic organization, producing 
one of its main products (named QDM). Based on a set of QDMs, SISDOT then derives another estimate 
of MEMs that should be assigned to a concrete organizational unity (a QDMP). 

Before generating a QDM, it is necessary to catalog all relevant MEMs to a given 
class of materials and then specify \callers (HLRs) that relate MEMs to the generic organizational 
unities, considering different elements of the structure---from the type of the unity and 
sub-unities to the qualifications of a soldier. That is, it is possible to elaborate a 
\shc that states that a military ambulance would be assigned to all \emph{drivers within medical units}. 
Differently, it is also possible to state that an automatic rifle will be assigned to all \emph{sergeant that has 
been qualified as a shooter}. There is some degree of flexibility for writing these 
rules, that might be from two different kinds: rules that consider the entire structure 
of the Brazilian Army and rules that refer to specific elements of the organizational 
structure. The former details rules that are more specific for a set of military qualifications. The second allows 
the generation of rules that might be reused through different unities. 

Deriving QDMs and QDMPs is not trivial and its complexity is due to several 
factors, including the rich organizational structure of the Brazilian Army, 
the huge number of existing functions and qualifications that must be 
considered, and the reasonable number of MEMs. The process might also vary 
depending on the class of the materials. Moreover, it is possible to specify the  
\callers in different ways, which in the end might interact to each other leading 
to undesired results (e.g., the same material being assigned several times 
to a given soldier). 

Due to number of logical decisions, using an imperative language to implement the process of QDM generation 
is not an interesting choice. However, the SIGELOG architecture constraints the development 
to use the Java Enterprise Edition platform---so we had little space to introduce a 
more suitable language (such as Prolog) to generate QDMs. Other constraints guided 
the architectural decisions of SISDOT, including:  

\begin{itemize}
\item Maintenance: the set of languages and libraries are limited to those already used in the SIGELOG development.  
\item Performance: considering a total of 1000 \callers, the system must be able to generate a QDM in less than 5 seconds.
\item Testing: the technical decisions must not compromise the testability of the system, being expected a high degree of automated tests. 
\end{itemize}

To deal with the first constraint, we decided to use a meta-programming approach that derives 
\emph{low-level rules} for Drools (a rule based engine for Java) \emph{on the fly}, from a set of \callers represented 
as domain objects that must exist in a database. In this way, it is not necessary to understand 
the language used to specify rules in Drools. Surely, although this approach increases flexibility and  
allows us to externalize the \emph{low-level rules}, it might introduce some undesired effect on 
the performance of the system: we generate all low-level rules every time a QDM is built. We are using 
a cache mechanism and performance tests (as we discuss in Section~\ref{sec:case_study}) 
shows that our architecture fulfills the second constraint. For the third constraint, we implemented a DSL that simplifies the specification and 
execution of test cases. 

In the next sections (Section~\ref{sec:rbs} and Section~\ref{sec:gp}) 
we introduce some technical background that might help the reader to understand 
the design and architectural decisions we take while implementing SISDOT.  
 
%% \subsection{Software Architecture}

%% % Arquitetura de software se tornou mais ampla e complexa devido às características dos sistema atuais, tais como: distribuídos e interconectados, escaláveis horizontalmente e prontos para um ambiente de computação em nuvem, publicados automaticamente em ambientes diversos, atualizáveis com mínimo de downtime, resilientes e adaptativos \cite{hohpe2016}. Há também uma diferença em relação à abordagem tradicional, que trata a arquitetura como um conjunto de componentes e conectores, ao tratar a arquitetura como um conjunto de decisões arquiteturais realizadas no sistema \cite{bosch2004}, \cite{jansen2005}.

%% Software architecture has become more extensive and complex due to the characteristics of the current systems, such as: distributed and interconnected, scalable horizontally and ready for a cloud computing environment, automatically published in diverse, updatable environments with minimal downtime, resilient and adaptive \cite{hohpe2016}. There is also a difference from the traditional approach, which treats architecture as a set of components and connectors, by treating architecture as a set of architectural decisions made in the system \cite{bosch2004}, \cite{jansen2005}.

%% % Arquitetura de software é um conjunto de estruturas necessárias para raciocinar sobre um sistema, que compreendem elementos de software, relações entre eles e propriedades de ambos \cite{bass2013software}. Existem três categorias de estruturas arquiteturais \cite{clements2011documenting}, \cite{bass2013software}. A arquitetura de software está relacionada diretamente aos atributos de qualidade do sistema; permite raciocinar sobre o impacto de mudanças enquanto o sistema evolui; a documentação da arquitetura melhora a comunicação entre os interessados e serve como base de treinamento de novos membros da equipe; serve como uma receita para a implementação do sistema; ajuda no planejamento de custos e cronograma de atividades; possibilita o reuso de soluções \cite{bass2013software}. 

%% Software architecture is a set of structures necessary to reason about a system, which comprise elements of software, relations between them and properties of both \cite{bass2013software}. There are three categories of architectural structures \cite{clements2011documenting}, \cite{bass2013software}. The software architecture is directly related to the quality attributes of the system; it allows to reason about the impact of changes as the system evolves; architecture documentation improves communication among stakeholders and serves as a training base for new team members; serves as a recipe for the implementation of the system; help with cost planning and schedule of activities; enables the reuse of solutions \cite{bass2013software}.

%% % Os atributos de qualidade considerados relevantes para a arquitetura proposta neste trabalho são: 

%% The quality attributes considered relevant for the architecture proposed in this work are:

%% \begin{itemize}

%% % \item Simplicidade na definição das regras de distribuição de materiais, transformando definições de alto nível  em definições baseadas em regras de motor de inferência, o que facilita a manutenção e extrai tais definições do código fonte de uma aplicação. Isso motivou o uso de programação generativa (em termos de meta programação) e um motor de inferência específico para a linguagem Java;

%% \item Simplicity in defining material distribution rules by transforming high-level definitions into definitions based on inference engine rules, which facilitates maintenance and extracts such definitions from the source code of an application. This motivated the use of generative programming (in terms of meta programming) and an inference engine specific to the Java language;

%% % \item Facilidade para execução de testes automatizados, motivando a definição de uma Domain-Specific Languages (DSL) para descrever conceitos relacionados à dotação de materiais para o Exército Brasileiro. A partir DSL, são gerados casos de testes baseados em ``especificações executáveis'' e classes que instanciam objetos do domínio. 

%% \item Ease of execution of automated tests, motivating the definition of a Domain-Specific Languages (DSL) to describe concepts related to the endowment of materials for the Brazilian Army. From DSL, test cases are generated based on ``executable specifications'' and classes that instantiate objects of the domain.

%% \end{itemize}


%\section{Technical Background} 

\section{Rule Based Systems}
\label{sec:rbs}

% Uma regra de negócio é uma declaração compacta, atômica e bem formada sobre um aspecto do negócio que pode ser expresso em termos que podem estar diretamente relacionados com o negócio e seus colaboradores, usando linguagem simples e não ambígua que é acessível a todas as partes interessadas: proprietário da empresa, analista de negócios, arquitetos, técnicos, cliente, dentre outros \cite{graham2007}. Esse tipo de estrutura é muito importante para as organizações, já que elas precisam lidar cada vez mais com  cenários complexos. Estes cenários são compostos por um grande número de decisões individuais, que trabalham em conjunto para fornecer uma avaliação complexa da organização como um todo \cite{salatino2016}.

The process for deriving QDMs consists of applying a set of business rules that 
relates MEMs to the organizational structure of the Brazilian Army. In this paper, we named 
this set of business rules as \callers. A business rule is a compact, atomic, and well-formed 
statement about an aspect of the business that can be expressed in terms that may be 
directly related to the business and its employees using a simple and unambiguous language 
that might be accessible to all stakeholders~\cite{graham2007business}. This type of structure is very important for organizations, 
as they need to deal more and more with complex scenarios. These scenarios consist of a 
large number of individual decisions, working together to provide a 
complex assessment of the organization as a whole \cite{salatino2016mastering}.

% Sistemas baseados em regras (Rule Based Systems -- RBS), também conhecidos como production systems ou expert systems, são a forma mais simples de inteligência artificial, usando regras como representação do conhecimento \cite{grosan2011}. Ao invés de representar o conhecimento de forma declarativa e estática como um conjunto de coisas que são verdadeiras, RBS representa o conhecimento em termos de um conjunto de regras que diz o que fazer ou o que concluir em diferentes situações \cite{grosan2011}. Essas regras são também conhecidas como: condição-ação, produções, situação-ação ou se-então \cite{russell2010}.

Rule Based Systems (RBS), also known as production systems or expert systems, are the 
simplest form of artificial intelligence that help to design solutions that reason about 
and take actions from business rules (a feature that is interesting in the context of SISDOT), using rules such as knowledge representation~\cite{grosan2011}.Instead of representing knowledge in a declarative and static way as a set of things that are true, 
RBS represents knowledge in terms of a set of rules that tells what to do or what to accomplish in different 
situations~\cite{grosan2011}. 
RBS consist of two basic elements: a \emph{knowledge base} to store knowledge and an \emph{inference engine} 
that implements  algorithms to manipulate the knowledge represented in the 
knowledge base~\cite{grosan2011,DBLP:books/daglib/0070547,gallacher1989}.
The knowledge base stores facts and rules~\cite{DBLP:journals/cacm/Hayes-Roth85}. a fact is a usually static statement about properties, relations, 
or propositions~\cite{DBLP:journals/cacm/Hayes-Roth85} and should be something relevant to the initial state of the system~\cite{grosan2011}. 
A rule is a conditional statement that links certain conditions to actions or results \cite{abraham2005}, being able to express 
policies, preferences, and constraints. Rules can be used to express deductive knowledge, 
such as logical relationships, and thus support inference, verification, or evaluation tasks~\cite{DBLP:journals/cacm/Hayes-Roth85}.

A rule ``if-then'' takes a form like \emph{``if x is A then take a set of actions''}. 
The conditional part is known as antecedent, premise, condition, or left-hand-side (LHS). The other part 
is known as consequent, conclusion, action, or right-hand-side (RHS)~\cite{grosan2011,abraham2005}. 
Rule-based systems works as follows~\cite{grosan2011}: it starts with a knowledge base, 
which contains all appropriate knowledge coded using ``if-then'' rules, and a working memory, 
which contain all data, assertions, and information initially known. 

The system examines all 
rule conditions (LHS) and determines a subset of rules whose conditions are satisfied, based on the data 
available at the working memory. From this information set, some of the rules are triggered. 
The choice of rules is based on a conflict resolution strategy. When the rule is triggered, 
all actions specified in its \texttt{then} clause (RHS) are executed. These actions can modify the 
working memory, the rule base itself, or take whatever action that the system programmer chooses to include. 
This \emph{trigger rule loop} continues until a termination criterion is met. This end criterion can be 
given by the fact that there are no more rules whose conditions are satisfied or a rule 
is triggered whose action specifies that the program should be finalized.

Drools is an example of a business logic integration platform (BLiP) written in the Java programming language and 
that fits some requirements of SISDOT (in particular, its integration with Java and the Wildfly application 
server). We use Drools as the \emph{rule-based system} in the QDM generation. However, to abstract the use 
of a RBS, we generate the Drools rules using a meta-programming approach, a specific technique for 
\emph{generative programming}. 

\section{Generative Programming}
\label{sec:gp}

Generative Programming (GP) is related to the design and implementation of 
software components that can be combined to generate specialized and highly 
optimized systems, fulfilling specific requirements~\cite{DBLP:phd/dnb/Czarnecki99}. 
With this approach, instead of developing solutions from scratch, domain-specific 
solutions are generated from reusable software components \cite{arora2009}. 
GP is applied on the manufacturing of software products from components in a system 
in an automated way (and within economic constraints), i.e., using an approach 
that is similar to that found in industries that have been 
producing mechanical, electronic, and other goods for decades \cite{barth2002}.
The main idea in GP is similar to the one that led people to switch from Assembly 
languages to high-level languages. 

In our context, we use GP to deal with two different aspects of SISDOT. First, 
we raise the level of abstraction by automatically generating low-level Drools 
rules from an existing set of \callers (previously defined by the domain experts). 
This is a type of \emph{meta-programming} based on template engines---we use a 
template to generate a set of facts and rules (which are the fundamental mechanisms of 
logic programming) from domain objects. We also raise the level of abstraction to 
assist developers in the specification and execution of test cases, using a 
domain specific language and a set of customized libraries and tools.  


\subsection{Metaprogramming and Template Engines}

Metaprogramming concerns to the implementation of programs that 
generate other programs. In the context of SISDOT, we use a metaprogramming 
technique to generate programs that are interpreted by Drools. The main goal 
is to abstract the use of a specific language for implementing the 
\emph{low-level rules} used to generate QDMs. There are many different 
approaches for metaprogramming, including preprocessing, source to source 
transformations, and template engines. 
A template engine is a generic tool for generating textual output from templates and data 
files. They can be used 
in the development of software that requires the automatic generation of code 
according to specific purposes~\cite{benato2017}. 

The need for dynamically generated web pages has led to the development of numerous template engines 
in an effort to make web application development easier, improve flexibility, 
reduce maintenance costs, and enable parallel encoding and development using HTML like languages. 
In addition to being widely used in dynamic web page generation, templates are used in other tasks, 
being one of the main techniques to generate source code from high level specifications. More 
recently, languages such as Clojure, Ruby, Scala, and Haskell adopted the use of template and \emph{quasiquote} 
mechanisms as a strategy for code reuse and program generation (some of them considering type systems).
A template is a text document that 
combines placeholders and linguistic formulas used to describe something in a specific domain~\cite{segura2017}. 

In this work we use the FreeMarker template engine, which is an open-source software designed to 
generate text from templates~\cite{radjenovic2009}. FreeMarker was chosen for several reasons, 
such as: it follows a general purpose model; it is faster than other template engines; it supports 
shared variables and model loaders; and its templates can be loaded or reloaded at runtime without the 
need to (re)compile the application~\cite{benato2017,parr2006}.

\subsection{Domain Specific Languages}

A language is a set of valid sentences, serving as a mechanism for expressing intentions \cite{parr2010}. 
Creating a new language is a time-consuming task, requires experience, and is usually performed by engineers specialized 
in languages construction~\cite{karsai2014}. To implement a language, one must construct an application that reads sentences and reacts 
appropriately to the phrases and input symbols it discovers.
The need for new languages for many growing domains is increasing, as well as the emergence of more sophisticated 
tools that allow software engineers to define a new language with reasonable effort. As a result, a growing number of DSLs 
are being developed to increase developer productivity within specific domains~\cite{karsai2014}. 


Domain Specific Languages (DSL) are specification languages or programming languages with high level of abstraction, 
simple and concise \cite{raja2010}, focused on specific domains, and are designed to facilitate the construction of applications, 
usually declaratively, with limited expressiveness lines of code, which solve these specific problems \cite{neeraj2017}. 
A DSL is a computer language of limited expressiveness, through appropriate notations and abstractions, focused and generally 
restricted to a specific problem domain \cite{fowler2013,vanDeursen2000}. DSLs have the 
potential to reduce the complexity of software development by increasing the level of abstraction for a domain. 
According to the application domain, different notations (textual, graphical, tabular) are used \cite{pfeiffer2008}.


In this work we used Xtext ~\cite{eysholdt2010} (a language workbench) to implement a DSL that helps
developers to specify and execute test cases related to the process of QDM generation.
The main rationale to building this DSL was the significant costs related to specifying \callers
(particularly during acceptance tests). In the next section we present more details about
our architectural decisions and implementation of our metaprogramming and DSL approach,
which deals with the generation and test of QDMs. 

