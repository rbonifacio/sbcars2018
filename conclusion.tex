\section{Conclusion}
\label{sec:conclusao}

O objetivo desse trabalho foi descrever as principais decisões arquiteturais e de design adotadas para implementar os mecanismos para derivar regras de negócio de alto nível em regras específicas para o motor de regras Drools e a definição de uma DSL que facilita a declaração de regras de negócio de alto nível. Além da apresentação do início do trabalho para geração automática de testes para complementar os testes definidos manualmente.

Foi realizado um estudo de caso para validar a arquitetura proposta a partir da avaliação do principal caso de uso do SISDOT, relacionado à geração de QDMs. A partir dos resultados obtidos com o estudo de caso foi possível observar uma diminuição considerável, de 45\% em média, da quantidade de linhas de código necessárias para a declaração de um chamador usando a DSL criada em comparação com a representação do mesmo chamador em Java, indicando um possível ganho de produtividade para o testador. Outro resultado do estudo de caso demonstrou a aderência do sistema ao requisito funcional declarado, de limitar o tempo de geração de um QDM em 5 segundos. Esse limite não foi ultrapassado nem quando foi utilizado o dobro da quantidade de chamadores esperados para o sistema em produção.

Although our architecture target the specific needs of the Brazilian Army, we believe that logistic systems from other institutions might benefit from our decisions.

Para refinar os resultados obtidos com a arquitetura proposta foram definidos os os seguintes trabalhos futuros:

\begin{itemize}
\item Refactoring dos serviços de QDM para otimizar a sua geração;
\item Finalizar a geração de testes automáticos de forma a responder à RQ.3;
\item Coletar e analisar as métricas novamente quando houver uma quantidade expressiva de chamadores, os quais continuam em constante criação pelo Exército Brasileiro, com o intuito de testar a arquitetura proposta.

\end{itemize}